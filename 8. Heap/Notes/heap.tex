\section{Heap Data Structure: An In-Depth Overview}

\subsection{What Is a Heap?}

A \textbf{heap} is a specialized tree-based data structure that satisfies the \textbf{heap property}:

\begin{itemize}
    \item \textbf{Complete Binary Tree}: All levels are fully filled except possibly the last, which is filled from left to right.
    \item \textbf{Heap Property}:
    \begin{itemize}
        \item \textbf{Max-Heap}: Each parent node is greater than or equal to its child nodes.
        \item \textbf{Min-Heap}: Each parent node is less than or equal to its child nodes.
    \end{itemize}
\end{itemize}

Heaps are commonly implemented as arrays, where the parent-child relationships are determined by index calculations:
\begin{itemize}
    \item \textbf{Parent Index}: $\left\lfloor \frac{i - 1}{2} \right\rfloor$
    \item \textbf{Left Child Index}: $2 \cdot i + 1$
    \item \textbf{Right Child Index}: $2 \cdot i + 2$
\end{itemize}

\subsection{Core Operations}

\subsubsection{Insertion (Bubble Up / Heapify Up)}

To insert an element:

\begin{enumerate}
    \item Add the element at the end of the heap (maintaining the complete binary tree property).
    \item Compare the added element with its parent; if the heap property is violated, swap them.
    \item Repeat step 2 until the heap property is restored.
\end{enumerate}

This operation ensures that the smallest (or largest) element percolates to the top in a min-heap (or max-heap).

\subsubsection{Removal (Sink Down / Heapify Down)}

To remove the root element:

\begin{enumerate}
    \item Replace the root with the last element in the heap.
    \item Remove the last element.
    \item Compare the new root with its children; if the heap property is violated, swap it with the appropriate child.
    \item Repeat step 3 until the heap property is restored.
\end{enumerate}

This process ensures that the heap remains valid after the removal of the root.

\subsubsection{Peek}

Retrieve the root element (minimum in a min-heap or maximum in a max-heap) without removing it.

\subsection{Heap Applications}

\subsubsection{Priority Queues}

Heaps are ideal for implementing priority queues, where elements with higher priority are served before others.

\begin{itemize}
    \item \textbf{Efficiency}: Insertion and removal operations have a time complexity of $O(\log n)$, and peeking has a time complexity of $O(1)$.
\end{itemize}

\subsubsection{Heap Sort}

Heap sort is a comparison-based sorting technique that uses a binary heap data structure.

\textbf{Algorithm}:
\begin{enumerate}
    \item Build a max-heap from the input data.
    \item At this point, the largest item is stored at the root of the heap.
    \item Replace it with the last item of the heap followed by reducing the size of the heap by one.
    \item Heapify the root of the tree.
    \item Repeat steps 3–4 until the size of the heap is greater than 1.
\end{enumerate}

\textbf{Time Complexity}: $O(n \log n)$

\textbf{Space Complexity}: $O(1)$ (in-place sorting)

\subsubsection{Graph Algorithms}

Heaps are used in algorithms like Dijkstra's and Prim's for efficiently selecting the next vertex with the smallest (or largest) key.

\subsection{Time and Space Complexity}

\begin{tabular}{|c|c|}
\hline
\textbf{Operation} & \textbf{Time Complexity} \\
\hline
Insertion & $O(\log n)$ \\
Deletion  & $O(\log n)$ \\
Peek      & $O(1)$ \\
Search    & $O(n)$ \\
\hline
\end{tabular}

\textbf{Space Complexity}: $O(n)$, where $n$ is the number of elements in the heap.
