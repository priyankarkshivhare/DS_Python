\section{Graphs}

\subsection{What is a Graph?}
A graph is a non-linear data structure consisting of \textbf{vertices (nodes)} and \textbf{edges (connections)}. Graphs can be:
\begin{itemize}
    \item \textbf{Directed} (edges have a direction)
    \item \textbf{Undirected} (bidirectional edges)
\end{itemize}
Edges may also have \textbf{weights} in weighted graphs.

\subsection{Representing a Graph}
Common representations:
\begin{itemize}
    \item \textbf{Adjacency List}: Best for sparse graphs. Each vertex stores a list of its neighbors.
    \item \textbf{Adjacency Matrix}: Good for dense graphs, but can waste space.
\end{itemize}

\subsection{Summary of Key Methods}
\begin{tabular}{|c|c|c|}
    \hline
    \textbf{Method} & \textbf{Purpose} & \textbf{Time Complexity} \\
    \hline
    \texttt{add\_vertex} & Adds a new vertex & \textbf{O(1)} \\
    \hline
    \texttt{add\_edge} & Adds a bidirectional edge & \textbf{O(1)} \\
    \hline
    \texttt{remove\_edge} & Removes a bidirectional edge & \textbf{O(V)} \\
    \hline
    \texttt{remove\_vertex} & Removes a vertex and its connections & \textbf{O(V + E)} \\
    \hline
    \texttt{print\_graph} & Prints the adjacency list & \textbf{O(V + E)} \\
    \hline
\end{tabular}

